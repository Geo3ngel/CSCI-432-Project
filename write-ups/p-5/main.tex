\documentclass{article}
\usepackage[english]{babel}
\usepackage[utf8]{inputenc}
\usepackage{fancyhdr}
\usepackage{fasy-hw}
\pagestyle{fancy}
\fancyhf{}
\lhead{Project P-5: Demonstration of Progress}

\rhead{Troy Oster, George Engel, Gavin Austin, Thomas Herndon}

\begin{document}
   \section{What we have done so far}
   
   \subsection{Algorithm Selection}
   The first thing that was done to complete this project was to select the algorithm we wanted to research. To do this, we started by researching the different types of algorithms that existed, limiting ourselves to algorithms to those of which were invented after 1980. After some research, we narrowed the choices down to two or three different algorithms. After some further deliberation, we decided to use the Burrows-Wheeler transform for this project. We chose the Burrows-Wheeler transform because we found the algorithm to be both interesting and applicable to many relevant topics in computer science.
   
   \subsection{Identifying the +1 Element}
   To complete the +1 element of the project, we decided to implement the algorithm and walk through an example of it functioning correctly in the form of a video. This +1 element was selected to effectively demonstrate the functionality of the algorithm, as well as to illustrate our implementation. The video portion of the +1 element will also describe the possible use cases for the algorithm.
   
   \subsection{Completing the +1 Element}
   To implement the Burrows-Wheeler transform, we first began to read documentation on the algorithm, as well as other implementations of the algorithm. From there, we decided to move on to the implementation phase. For this, we chose to use Python as it is the language we were most comfortable writing the algorithm in. To make the algorithm better for demonstration purposes, we added a stepper function that waits for user input to move forward with the next step of the algorithm. This stepper function allows for a more detailed output at each step of the process. It also allows us to demonstrate the algorithm implementation by hand side-by-side at each step of the algorithm in our video.
   
   \section{What is left to be done}
   
   \subsection{Video: Script Writing}
   To get ready to record the video, we decided to write a script. This script includes everything we intend to demonstrate about the Burrows-Wheeler transform. The script in this stage is mainly a bulleted list of what we plan to say in our video.
   
   \subsection{Video: Recording}
   At this stage, we are planning to have Troy do the voice over for the video demonstration of the Burrows-Wheeler transform. In the video, that we have yet to record, we plan to show the code and the output of our implementation of the Burrows-Wheeler transform.
   
      \section{References}
   \newline \makebox[.5cm]{[1]} Adjeroh, Donald, et al. The Burrows-Wheeler Transform: Data Compression, Suffix Arrays, and Pattern Matching. Springer, 2011.\par
    \makebox[.5cm]{[2]}  Burrows, Michael; Wheeler, David J. (1994), A block sorting lossless data compression algorithm, Technical Report 124, Digital Equipment Corporation\par
    \makebox[.5cm]{[3]} Heng Li, Richard Durbin, Fast and accurate short read alignment with Burrows–Wheeler transform, Bioinformatics, Volume 25, Issue 14, 15 July 2009, Pages 1754–1760, https://doi.org/10.1093/bioinformatics/btp324\par
    \makebox[.5cm]{[4]} Oster, Troy; Engel, George; Austin, Gavin; Herndon, Thomas. (2019). Project P-3. Unpublished project part. Montana State University.\par
\end{document}