\documentclass{article}
\usepackage[english]{babel}
\usepackage[utf8]{inputenc}
\usepackage{fancyhdr}
\usepackage{fasy-hw}

\pagestyle{fancy}
\fancyhf{}
\lhead{Project P-5: Demonstration of Progress}

\rhead{Troy Oster, George Engel, Gavin Austin, Thomas Herndon}

\begin{document}
   \section{What we have done so far}
   \subsection{Algorithm Selection}
   The first thing that was done to complete this project was to select the algorithm we wanted to research. To do this, we started by researching the different types of algorithms that existed, limiting ourselves to algorithms to those of which were invented after 1980. After some research, we narrowed the choices down to two or three different algorithms. After some further deliberation, we decided to use the Burrows-Wheeler transform for this project. We chose the Burrows-Wheeler transform because we found the algorithm to be both interesting and applicable to many relevant topics in computer science.
   
   \subsection{Identifying the +1 Element}
   To complete the +1 element of the project, we decided to implement the algorithm and walk through an example of it functioning correctly in the form of a video. This +1 element was selected to effectively demonstrate the functionality of the algorithm, as well as to illustrate our implementation. The video portion of the +1 element is specifically designed to highlight the functionality of the algorithm and to describe the possible use cases for the algorithm.
   \section{What is left to be done}
\end{document}